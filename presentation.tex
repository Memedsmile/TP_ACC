\documentclass[french]{beamer}
\usepackage{beamerthemesplit}
\usepackage[utf8]{inputenc, fontenc}
\usepackage[T1]{fontenc}
\usepackage[french]{babel}
\usepackage{times}
\usepackage{tabularx}
\usepackage{tkz-graph}
\usepackage{graphics}
\usepackage{tikz}
\usepackage{pgfplots}


\mode<presentation>
{
  \usetheme{Warsaw}
  \usefonttheme[onlysmall]{structurebold}
}

\beamertemplatetransparentcovered
\beamerboxesdeclarecolorscheme{myalert}{red}{black!5!averagebackgroundcolor}
\beamerboxesdeclarecolorscheme{mybox}{blue}{black!5!averagebackgroundcolor}

\usepackage{pgf,pgfarrows,pgfnodes,pgfautomata,pgfheaps,pgfshade}


\font\bbfnt=msbm6
\def\bbN{\mbox{\bbfnt N}}
\def\bbR{\mbox{\bbfnt R}}

\DeclareGraphicsExtensions{.pdf,.pdftex_t,.png,.jpg,.gif} 
\graphicspath{{./}{../Images/}} 

%===================================================
%===================================================

\title{Projet pratique Algorithme / Complexité / Calculabilité}
\author{Jean-Marc Lagniez, Viktor Lesnyak, Pierre-Alexandre Cimbe, Ahmed Rafik}
\institute{Master Informatique - Université Montpellier II}
\date{2013}

%===================================================
%===================================================

\begin{document}

\frame{\titlepage}

%===================================================
\section<presentation>*{Plan}
%===================================================

\note{L'exopsé est composé de trois parties, une partie présenation des algorithmes utilisés,
 une partie expérimentation et performances et une partie démonstration du programme}

%----------------------------------------
\begin{frame}
  \frametitle{Plan}
  \tableofcontents[part=1,pausesections]
\end{frame}

%----------------------------------------
% cette partie entraine l'affichage du plan à chaque nouveau début de section
\AtBeginSection[]
{
  \begin{frame}<beamer>
    \frametitle{Plan}
    \tableofcontents[current,currentsection]
  \end{frame}
}


%=================================================
\part<presentation>{Les Algorithmes utilisés}
%===================================================
\section[Les algorithmes étudiés]{Les algorithmes étudiés}
%===================================================

%+++++++++++++++++++++++++++++++++++++++++++++++++++
\subsection[Ford-Fulkerson]{Ford-Fulkerson}
%---------------------------------------------------

%--------------------------------------------------Graphe Init
\frame{
\frametitle{Algo Ford-Fulkerson}


\begin{columns}[c]
\begin{column}{7cm}
\begin{block}{Graphe initiale}
 
\usetikzlibrary{arrows}
\begin{flushleft}
\begin{tikzpicture}[>=latex']
 \SetUpEdge[lw         = 1pt,
            color      = orange,
            labelcolor = red!30,
            labelstyle = {draw,sloped}]
  \tikzset{node distance = 2cm}
  \GraphInit[vstyle=Normal]
  \Vertex[x=-3, y=3]{S}
  \Vertex[x=0, y=3]{A}
  \Vertex[x=3, y=3]{D}
  \Vertex[x=3, y=0]{T}
  \Vertex[x=0, y=0]{C}
  \Vertex[x=-3, y=0]{B}
  \tikzset{EdgeStyle/.style={->}}
  \Edge[label=$3$](S)(B)
  \Edge[label=$2$](S)(C)
  \Edge[label=$3$](S)(A)
  \Edge[label=$2$](C)(T)
  \Edge[label=$1$](D)(T)
  \Edge[label=$4$](A)(D)
  \Edge[label=$1$](C)(A)
  \Edge[label=$1$](D)(C)
  \Edge[label=$1$](B)(C)
  \tikzset{EdgeStyle/.style={->}, bend right}
  \Edge[label=${0,2}$](B)(T)
\end{tikzpicture}
\end{flushleft}
\end{block} 
\end{column}
  
  \begin{column}{3cm}
  \begin{block}{Graphe initiale}

  Soit G = (V,E) un graphe,avec
  V-ensemble des arcs et E-
  ensemble des sommets.
  \end{block}
  \end{column}
  \end{columns}  


}


%--------------------------------------------------Graphe Ecart avec flot null
\frame{
\frametitle{Algo Ford-Fulkerson}

\begin{columns}[c]
\begin{column}{8cm}
\begin{block}{Graphe d'ecart}


\usetikzlibrary{arrows}
\begin{flushleft}
\begin{tikzpicture}[>=latex']
 \SetUpEdge[lw         = 1pt,
            color      = orange,
            labelcolor = red!30,
            labelstyle = {draw,sloped}]
  \tikzset{node distance = 2cm}
  \GraphInit[vstyle=Normal]
  \Vertex[x=-3, y=3]{S}
  \Vertex[x=0, y=3]{A}
  \Vertex[x=3, y=3]{D}
  \Vertex[x=3, y=0]{T}
  \Vertex[x=0, y=0]{C}
  \Vertex[x=-3, y=0]{B}
  \tikzset{EdgeStyle/.style={->}}
  \Edge[label=${0,3}$](S)(B)
  \Edge[label=${0,2}$](S)(C)
  \Edge[label=${0,2}$](C)(T)
  \Edge[label=${0,1}$](C)(A)
  \Edge[label=${0,1}$](D)(C)
  \Edge[label=${0,1}$](B)(C)
  \Edge[label=${0,3}$](S)(A)
  \Edge[label=${0,4}$](A)(D)
  \Edge[label=${0,1}$](D)(T)
  \tikzset{EdgeStyle/.style={->}, bend right}
  \Edge[label=${0,2}$](B)(T)
  \draw[orange](3,0)--(4,4);
  \draw[orange,->](4,4)--(-3,3);
  \node at (4.25,4.25){0,$\infty$};
  

\end{tikzpicture}
\end{flushleft}

\end{block} 
\end{column}
  
  \begin{column}{3cm}
  \begin{block}{Graphe d'ecart}
  Pour passer de notre graphe G
  au Graphe d'ecart G$_e$ on 
  applique un flot null sur 
  toutes les arcs et on ajout
  un arc qui va de la source(S)
  vers le puit(T).
  \end{block}
  \end{column}
  \end{columns}  

}



%--------------------------------------------------Graphe Ecart chemin ameliorant
\frame{
\frametitle{Algo Ford-Fulkerson}

\begin{columns}[c]
\begin{column}{8cm}
\begin{block}{Chemin ameliorant}

\usetikzlibrary{arrows}
\begin{flushleft}
\begin{tikzpicture}[>=latex']
 \SetUpEdge[lw         = 1pt,
            color      = orange,
            labelcolor = red!30,
            labelstyle = {draw,sloped}]
  \tikzset{node distance = 2cm}d
  \GraphInit[vstyle=Normal]
  \Vertex[x=-3, y=3]{S}
  \Vertex[x=0, y=3]{A}
  \Vertex[x=3, y=3]{D}
  \Vertex[x=3, y=0]{T}
  \Vertex[x=0, y=0]{C}
  \Vertex[x=-3, y=0]{B}

  
 \tikzset{EdgeStyle/.style={->}}
  \Edge[label=${0,3}$](S)(B)
  \Edge[label=${0,2}$](S)(C)
  \Edge[label=${0,2}$](C)(T)
  \Edge[label=${0,1}$](C)(A)
  \Edge[label=${0,1}$](D)(C)
  \Edge[label=${0,1}$](B)(C)
  \tikzset{EdgeStyle/.style={->, color = red}}
  \Edge[label=${0,3}$](S)(A)
  \Edge[label=${0,4}$](A)(D)
  \Edge[label=${0,1}$](D)(T)
  \tikzset{EdgeStyle/.style={->, color = orange}, bend right}
  \Edge[label=${0,2}$](B)(T)
  \draw[orange](3,0)--(4,4);
  \draw[orange,->](4,4)--(-3,3);
  \node at (4.25,4.25){0,$\infty$};
  
\end{tikzpicture}
\end{flushleft}

\end{block} 
\end{column}
  
  \begin{column}{3cm}
  \begin{block}{Chemin ameliorant}
    Ensuite on choisi un chemin 
    ammeliorant sur le graphe
    d'ecart obtenue grace a 
    un parcour en largeur.
  \end{block}
  \end{column}
  \end{columns}  
}

%---------------------------------------------Mise a jour du flot

\frame{
\frametitle{Algo Ford-Fulkerson}

\begin{columns}[c]
\begin{column}{8cm}
\begin{block}{Chemin ameliorant}

\usetikzlibrary{arrows}
\begin{flushleft}
\begin{tikzpicture}[>=latex']
 \SetUpEdge[lw         = 1pt,
            color      = orange,
            labelcolor = red!30,
            labelstyle = {draw,sloped}]
  \tikzset{node distance = 2cm}
  \GraphInit[vstyle=Normal]
  \Vertex[x=-3, y=3]{S}
  \Vertex[x=0, y=3]{A}
  \Vertex[x=3, y=3]{D}
  \Vertex[x=3, y=0]{T}
  \Vertex[x=0, y=0]{C}
  \Vertex[x=-3, y=0]{B}

  
 \tikzset{EdgeStyle/.style={->}}
  \Edge[label=${0,3}$](S)(B)
  \Edge[label=${0,2}$](S)(C)
  \Edge[label=${0,2}$](C)(T)
  \Edge[label=${0,1}$](C)(A)
  \Edge[label=${0,1}$](D)(C)
  \Edge[label=${0,1}$](B)(C)
  \tikzset{EdgeStyle/.style={->, color = red}}
  \Edge[label=${1,3}$](S)(A)
  \Edge[label=${1,4}$](A)(D)
  \tikzset{EdgeStyle/.style={<-, color = red}}
  \Edge[label=${1,1}$](D)(T)
  \tikzset{EdgeStyle/.style={->, color = orange}, bend right}
  \Edge[label=${0,2}$](B)(T)
  \draw[orange](3,0)--(4,4);
  \draw[orange,->](4,4)--(-3,3);
  \node at (4.25,4.25){1,$\infty$};
  
\end{tikzpicture}
\end{flushleft}

\end{block} 
\end{column}
  
  \begin{column}{3cm}
  \begin{block}{Chemin ameliorant}
    En utilisant le flot le plus
    petit de ce chemin on met a
     jour le graphe d'ecart.
  \end{block}
  \end{column}
  \end{columns}  
}


%---------------------------------------------Graphe d'ecart finale

\frame{
\frametitle{Algo Ford-Fulkerson}

\begin{columns}[c]
\begin{column}{8cm}
\begin{block}{Graphe finale}

\usetikzlibrary{arrows}
\begin{flushleft}
\begin{tikzpicture}[>=latex']
 \SetUpEdge[lw         = 1pt,
            color      = orange,
            labelcolor = red!30,
            labelstyle = {draw,sloped}]
  \tikzset{node distance = 2cm}
  \GraphInit[vstyle=Normal]
  \Vertex[x=-3, y=3]{S}
  \Vertex[x=0, y=3]{A}
  \Vertex[x=3, y=3]{D}
  \Vertex[x=3, y=0]{T}
  \Vertex[x=0, y=0]{C}
  \Vertex[x=-3, y=0]{B}

  
 \tikzset{EdgeStyle/.style={->}}
  \Edge[label=${0,2}$](S)(C)
  \Edge[label=${0,1}$](C)(A)
  \Edge[label=${1,3}$](S)(A)
  \Edge[label=${1,4}$](A)(D)
  \Edge[label=${2,3}$](S)(A)
  \Edge[label=${2,4}$](A)(D)
  \tikzset{EdgeStyle/.style={<-}}
  \Edge[label=${1,1}$](D)(T)
  \Edge[label=${3,3}$](S)(B)
  \Edge[label=${1,1}$](B)(C)
  \Edge[label=${2,2}$](C)(T)
  \Edge[label=${1,1}$](D)(C)

  \tikzset{EdgeStyle/.style={<-, color = orange}, bend right}
  \Edge[label=${2,2}$](B)(T)
  \draw[orange](3,0)--(4,4);
  \draw[orange,->](4,4)--(-3,3);
  \node at (4.25,4.25){5,$\infty$};
  
\end{tikzpicture}
\end{flushleft}

\end{block} 
\end{column}
  
  \begin{column}{3cm}
  \begin{block}{Graphe finale}
    Une foi tout les chemin
    ameliorants sont parcouru,
    on obtien un graphe d'ecart
    complet avec le flot maximal
    (das notre cas c'est 5).
  \end{block}
  \end{column}
  \end{columns}  
}


%+++++++++++++++++++++++++++++++++++++++++++++++++++
\subsection[Edmonds-Karp]{Edmonds-Karp}
%---------------------------------------------------



%--------------------------------------------------Graphe initiale
\frame{
\frametitle{Algo Edmonds-Karp}
 

\begin{columns}[c]
\begin{column}{7cm}
\begin{block}{Graphe initiale}
 
\usetikzlibrary{arrows}
\begin{flushleft}
\begin{tikzpicture}[>=latex']
 \SetUpEdge[lw         = 1pt,
            color      = orange,
            labelcolor = red!30,
            labelstyle = {draw,sloped}]
  \tikzset{node distance = 2cm}
  \GraphInit[vstyle=Normal]
  \Vertex[x=-3, y=3]{S}
  \Vertex[x=0, y=3]{A}
  \Vertex[x=3, y=3]{D}
  \Vertex[x=3, y=0]{T}
  \Vertex[x=0, y=0]{C}
  \Vertex[x=-3, y=0]{B}
  \tikzset{EdgeStyle/.style={->}}
  \Edge[label=$3$](S)(B)
  \Edge[label=$2$](S)(C)
  \Edge[label=$3$](S)(A)
  \Edge[label=$2$](C)(T)
  \Edge[label=$1$](D)(T)
  \Edge[label=$4$](A)(D)
  \Edge[label=$1$](C)(A)
  \Edge[label=$1$](D)(C)
  \Edge[label=$1$](B)(C)
  \tikzset{EdgeStyle/.style={->}, bend right}
  \Edge[label=$2$](B)(T)
\end{tikzpicture}
\end{flushleft}
\end{block} 
\end{column}
  
  \begin{column}{3cm}
  \begin{block}{Graphe initiale}
  On reprends le meme
  graphe initiale.\\
  Soit G = (V,E) un graphe,avec
  V-ensemble des arcs et E-
  ensemble des sommets.
  \end{block}
  \end{column}
  \end{columns}  
  
}



%--------------------------------------------------Graphe d'ecart


\frame{
\frametitle{Algo Edmonds-Karp}

\begin{columns}[c]
\begin{column}{8cm}
\begin{block}{Graphe d'ecart}


\usetikzlibrary{arrows}
\begin{flushleft}
\begin{tikzpicture}[>=latex']
 \SetUpEdge[lw         = 1pt,
            color      = orange,
            labelcolor = red!30,
            labelstyle = {draw,sloped}]
  \tikzset{node distance = 2cm}
  \GraphInit[vstyle=Normal]
  \Vertex[x=-3, y=3]{S}
  \Vertex[x=0, y=3]{A}
  \Vertex[x=3, y=3]{D}
  \Vertex[x=3, y=0]{T}
  \Vertex[x=0, y=0]{C}
  \Vertex[x=-3, y=0]{B}
  \tikzset{EdgeStyle/.style={->}}
  \Edge[label=${0,3}$](S)(B)
  \Edge[label=${0,2}$](S)(C)
  \Edge[label=${0,2}$](C)(T)
  \Edge[label=${0,1}$](C)(A)
  \Edge[label=${0,1}$](D)(C)
  \Edge[label=${0,1}$](B)(C)
  \Edge[label=${0,3}$](S)(A)
  \Edge[label=${0,4}$](A)(D)
  \Edge[label=${0,1}$](D)(T)
  \tikzset{EdgeStyle/.style={->}, bend right}
  \Edge[label=${0,2}$](B)(T)
  \draw[orange](3,0)--(4,4);
  \draw[orange,->](4,4)--(-3,3);
  \node at (4.25,4.25){0,$\infty$};
  

\end{tikzpicture}
\end{flushleft}

\end{block} 
\end{column}
  
  \begin{column}{3cm}
  \begin{block}{Graphe d'ecart}
  On refait a nouveau a partir 
  de graphe G le Graphe d'ecart 
  G$_e$ on applique un flot null 
  sur toutes les arcs et on ajout
  un arc qui va de la source(S)
  vers le puit(T).
  \end{block}
  \end{column}
  \end{columns}  

}


%------------------------------------------graphe ecart chemin ameliorant


\frame{
\frametitle{Algo Edmonds-Karp}

\begin{columns}[c]
\begin{column}{8cm}
\begin{block}{Chemin ameliorant}

\usetikzlibrary{arrows}
\begin{flushleft}
\begin{tikzpicture}[>=latex']
 \SetUpEdge[lw         = 1pt,
            color      = orange,
            labelcolor = red!30,
            labelstyle = {draw,sloped}]
  \tikzset{node distance = 2cm}
  \GraphInit[vstyle=Normal]
  \Vertex[x=-3, y=3]{S}
  \Vertex[x=0, y=3]{A}
  \Vertex[x=3, y=3]{D}
  \Vertex[x=3, y=0]{T}
  \Vertex[x=0, y=0]{C}
  \Vertex[x=-3, y=0]{B}

  
 \tikzset{EdgeStyle/.style={->}}
  \Edge[label=${0,3}$](S)(B)
  \Edge[label=${0,1}$](C)(A)
  \Edge[label=${0,1}$](B)(C)
  \Edge[label=${0,3}$](S)(A)
  \Edge[label=${0,4}$](A)(D)
  \Edge[label=${0,1}$](D)(T)
  \Edge[label=${0,1}$](C)(D)
  \tikzset{EdgeStyle/.style={->, color = red}}
  \Edge[label=${0,2}$](S)(C)
  \Edge[label=${0,2}$](C)(T)
  \tikzset{EdgeStyle/.style={->, color = orange}, bend right}
  \Edge[label=${0,2}$](B)(T)
  \draw[orange](3,0)--(4,4);
  \draw[orange,->](4,4)--(-3,3);
  \node at (4.25,4.25){0,$\infty$};
  
\end{tikzpicture}
\end{flushleft}

\end{block} 
\end{column}
  
  \begin{column}{3cm}
  \begin{block}{Chemin ameliorant}
    On choisi ici un chemin ameliorant
    en fonction de plus court chemin,
    qui dans notre cas est calcule\\ 
    avec l'algo de Dijsktra
  \end{block}
  \end{column}
  \end{columns}

}



%-------------------------------------------------Mis a jour du graphe

\frame{
\frametitle{Algo Edmonds-Karp}

\begin{columns}[c]
\begin{column}{8cm}
\begin{block}{Chemin ameliorant}

\usetikzlibrary{arrows}
\begin{flushleft}

\begin{tikzpicture}[>=latex']
 \SetUpEdge[lw         = 1pt,
            color      = orange,
            labelcolor = red!30,
            labelstyle = {draw,sloped}]
  \tikzset{node distance = 2cm}
  \GraphInit[vstyle=Normal]
  \Vertex[x=-3, y=3]{S}
  \Vertex[x=0, y=3]{A}
  \Vertex[x=3, y=3]{D}
  \Vertex[x=3, y=0]{T}
  \Vertex[x=0, y=0]{C}
  \Vertex[x=-3, y=0]{B}

  
 \tikzset{EdgeStyle/.style={->}}
  \Edge[label=${0,3}$](S)(B)
  \Edge[label=${0,1}$](C)(A)
  \Edge[label=${0,1}$](B)(C)
  \Edge[label=${1,4}$](A)(D)
  \Edge[label=${1,3}$](S)(A)
  \Edge[label=${1,1}$](C)(D)
  \Edge[label=${1,1}$](D)(T)
  \tikzset{EdgeStyle/.style={<-, color = red}}
  \Edge[label=${2,2}$](S)(C)
  \Edge[label=${2,2}$](C)(T)
  \tikzset{EdgeStyle/.style={->, color = orange}, bend right}
  \Edge[label=${0,2}$](B)(T)
  \draw[orange](3,0)--(4,4);
  \draw[orange,->](4,4)--(-3,3);
  \node at (4.25,4.25){2,$\infty$};
  
\end{tikzpicture}
\end{flushleft}

\end{block} 
\end{column}
  
  \begin{column}{3cm}
  \begin{block}{Chemin ameliorant}
    En utilisant le flotle plus
    petit de ce chemin on met a
     jour le graphe d'ecart.
  \end{block}
  \end{column}
  \end{columns}  
}





%----------------------------------------Graphe d'ecart finale

\frame{
\frametitle{Algo Edmonds-Karp}

\begin{columns}[c]
\begin{column}{8cm}
\begin{block}{Graphe finale}

\usetikzlibrary{arrows}
\begin{flushleft}
\begin{tikzpicture}[>=latex']
 \SetUpEdge[lw         = 1pt,
            color      = orange,
            labelcolor = red!30,
            labelstyle = {draw,sloped}]
  \tikzset{node distance = 2cm}
  \GraphInit[vstyle=Normal]
  \Vertex[x=-3, y=3]{S}
  \Vertex[x=0, y=3]{A}
  \Vertex[x=3, y=3]{D}
  \Vertex[x=3, y=0]{T}
  \Vertex[x=0, y=0]{C}
  \Vertex[x=-3, y=0]{B}

  
 \tikzset{EdgeStyle/.style={->}}
  \Edge[label=${0,2}$](S)(C)
  \Edge[label=${0,1}$](C)(A)
  \Edge[label=${1,3}$](S)(A)
  \Edge[label=${1,4}$](A)(D)
  \Edge[label=${2,3}$](S)(A)
  \Edge[label=${2,4}$](A)(D)
  \tikzset{EdgeStyle/.style={<-}}
  \Edge[label=${1,1}$](D)(T)
  \Edge[label=${3,3}$](S)(B)
  \Edge[label=${1,1}$](B)(C)
  \Edge[label=${2,2}$](C)(T)
  \Edge[label=${1,1}$](D)(C)

  \tikzset{EdgeStyle/.style={<-, color = orange}, bend right}
  \Edge[label=${2,2}$](B)(T)
  \draw[orange](3,0)--(4,4);
  \draw[orange,->](4,4)--(-3,3);
  \node at (4.25,4.25){2,$\infty$};
  
\end{tikzpicture}
\end{flushleft}

\end{block} 
\end{column}
  
  \begin{column}{3cm}
  \begin{block}{Graphe finale}
    Une foi tout les chemin
    ameliorants sont parcouru,
    on obtien un graphe d'ecart
    complet avec le flot maximal
    (das notre cas c'est 5).
  \end{block}
  \end{column}
  \end{columns}  
}

%+++++++++++++++++++++++++++++++++++++++++++++++++++

\subsection[Dinic]{Dinic}
%---------------------------------------------------
%--------------------------------------------------
\frame{
\frametitle{AlgoD}
   
}

%+++++++++++++++++++++++++++++++++++++++++++++++++++

\subsection[Cacity Scaling]{Capacity Scaling}
%---------------------------------------------------
%--------------------------------------------------
\frame{
\frametitle{Algo Capacity Scaling}
   \begin{columns}[c]
\begin{column}{8cm}
\begin{block}{Graphe initial}


\usetikzlibrary{arrows}
\begin{flushleft}
\begin{center}
\begin{tikzpicture}[>=latex']
 \SetUpEdge[lw         = 1pt,
            color      = orange,
            labelcolor = red!30,
            labelstyle = {draw,sloped}]
  \tikzset{node distance = 2cm}
  \GraphInit[vstyle=Normal]
  \Vertex[x=-3, y=3]{S}
  \Vertex[x=0, y=3]{A}
  \Vertex[x=3, y=3]{D}
  \Vertex[x=3, y=0]{T}
  \Vertex[x=0, y=0]{C}
  \Vertex[x=-3, y=0]{B}
  \tikzset{EdgeStyle/.style={->}}
  \Edge[label=$3$](S)(B)
  \Edge[label=$2$](S)(C)
  \Edge[label=$3$](S)(A)
  \Edge[label=$2$](C)(T)
  \Edge[label=$1$](D)(T)
  \Edge[label=$4$](A)(D)
  \Edge[label=$1$](C)(A)
  \Edge[label=$1$](C)(D)
  \Edge[label=$1$](B)(C)
  \tikzset{EdgeStyle/.style={->}, bend right}
  \Edge[label=$2$](B)(T)
\end{tikzpicture}
\end{center}
\end{flushleft}

\end{block} 
\end{column}
  
  \begin{column}{3cm}
  \begin{block}{Initialisation}
  C=4;\\
  $\Delta=4$;\\
  F le flot max = 0;\\
  \end{block}
  \end{column}
  \end{columns}  
}

\frame{
\frametitle{Algo Capacity Scaling}
   \begin{columns}[c]
\begin{column}{8cm}
\begin{block}{G(f,4)}


\usetikzlibrary{arrows}
\begin{flushleft}
\begin{center}
\begin{tikzpicture}[>=latex']
 \SetUpEdge[lw         = 1pt,
            color      = orange,
            labelcolor = red!30,
            labelstyle = {draw,sloped}]
  \tikzset{node distance = 2cm}
  \GraphInit[vstyle=Normal]
  \Vertex[x=-3, y=3]{S}
  \Vertex[x=0, y=3]{A}
  \Vertex[x=3, y=3]{D}
  \Vertex[x=3, y=0]{T}
  \Vertex[x=0, y=0]{C}
  \Vertex[x=-3, y=0]{B}
  \tikzset{EdgeStyle/.style={->}}
  \Edge[label=$4$](A)(D)
  \tikzset{EdgeStyle/.style={->}, bend right}
\end{tikzpicture}
\end{center}
\end{flushleft}

\end{block} 
\end{column}
  
  \begin{column}{3cm}
  \begin{block}{Augmentation}
  Pas de chemin de s � p, $\Delta = \Delta/2 = 2$\\
  \end{block}
  \end{column}
  \end{columns}  
}

\frame{
\frametitle{Algo Capacity Scaling}
   \begin{columns}[c]
\begin{column}{8cm}
\begin{block}{G(f,2)}


\usetikzlibrary{arrows}
\begin{flushleft}
\begin{center}
\begin{tikzpicture}[>=latex']
 \SetUpEdge[lw         = 1pt,
            color      = orange,
            labelcolor = red!30,
            labelstyle = {draw,sloped}]
  \tikzset{node distance = 2cm}
  \GraphInit[vstyle=Normal]
  \Vertex[x=-3, y=3]{S}
  \Vertex[x=0, y=3]{A}
  \Vertex[x=3, y=3]{D}
  \Vertex[x=3, y=0]{T}
  \Vertex[x=0, y=0]{C}
  \Vertex[x=-3, y=0]{B}
  \tikzset{EdgeStyle/.style={->}}
  \Edge[label=$3$,color=green,labelcolor=green!30](S)(B)
  \Edge[label=$2$](S)(C)
  \Edge[label=$3$](S)(A)
  \Edge[label=$2$](C)(T)
  \Edge[label=$4$](A)(D)
  \tikzset{EdgeStyle/.style={->}, bend right}
  \Edge[label=$2$,color=green,labelcolor=green!30](B)(T)
\end{tikzpicture}
\end{center}
\end{flushleft}

\end{block} 
\end{column}
  
  \begin{column}{3cm}
  \begin{block}{Augmentation}
  Premier chemin :\\
$s \rightarrow b \rightarrow t$  \\
$\delta = 2$  \\
$F = 2$\\
  \end{block}
  \end{column}
  \end{columns}  
}

\frame{
\frametitle{Algo Capacity Scaling}
   \begin{columns}[c]
\begin{column}{8cm}
\begin{block}{G(f,2)}


\usetikzlibrary{arrows}
\begin{flushleft}
\begin{center}
\begin{tikzpicture}[>=latex']
 \SetUpEdge[lw         = 1pt,
            color      = orange,
            labelcolor = red!30,
            labelstyle = {draw,sloped}]
  \tikzset{node distance = 2cm}
  \GraphInit[vstyle=Normal]
  \Vertex[x=-3, y=3]{S}
  \Vertex[x=0, y=3]{A}
  \Vertex[x=3, y=3]{D}
  \Vertex[x=3, y=0]{T}
  \Vertex[x=0, y=0]{C}
  \Vertex[x=-3, y=0]{B}
  \tikzset{EdgeStyle/.style={->}}
  \Edge[label=$2$,color=green,labelcolor=green!30](S)(C)
  \Edge[label=$3$](S)(A)
  \Edge[label=$2$,color=green,labelcolor=green!30](C)(T)
  \Edge[label=$4$](A)(D)
  \tikzset{EdgeStyle/.style={->}, bend left}
  \Edge[label=$2$](T)(B)
  \Edge[label=$2$](B)(S)
  \Edge[label=$1$](S)(B)
\end{tikzpicture}
\end{center}
\end{flushleft}

\end{block} 
\end{column}
  
  \begin{column}{3cm}
  \begin{block}{Mise � jour du graphe d'�cart et Augmentation}
Second chemin :\\
$s \rightarrow c \rightarrow t$  \\
$\delta = 2$  \\
$F = 4$\\
  \end{block}
  \end{column}
  \end{columns}  
}

\frame{
\frametitle{Algo Capacity Scaling}
   \begin{columns}[c]
\begin{column}{8cm}
\begin{block}{G(f,2)}


\usetikzlibrary{arrows}
\begin{flushleft}
\begin{center}
\begin{tikzpicture}[>=latex']
 \SetUpEdge[lw         = 1pt,
            color      = orange,
            labelcolor = red!30,
            labelstyle = {draw,sloped}]
  \tikzset{node distance = 2cm}
  \GraphInit[vstyle=Normal]
  \Vertex[x=-3, y=3]{S}
  \Vertex[x=0, y=3]{A}
  \Vertex[x=3, y=3]{D}
  \Vertex[x=3, y=0]{T}
  \Vertex[x=0, y=0]{C}
  \Vertex[x=-3, y=0]{B}
  \tikzset{EdgeStyle/.style={->}}
  \Edge[label=$2$](C)(S)
  \Edge[label=$2$](T)(C)
  \Edge[label=$3$](S)(A)
  \Edge[label=$4$](A)(D)
   \tikzset{EdgeStyle/.style={->}, bend left}
  \Edge[label=$2$](T)(B)
  \Edge[label=$2$](B)(S)
\end{tikzpicture}
\end{center}
\end{flushleft}

\end{block} 
\end{column}
  
  \begin{column}{3cm}
  \begin{block}{Mise � jour du graphe d'�cart et Augmentation}
Pas d'autre chemin dans le graphe. $\Delta = \Delta/2 = 1$\\
  \end{block}
  \end{column}
  \end{columns}  
}

\frame{
\frametitle{Algo Capacity Scaling}
   \begin{columns}[c]
\begin{column}{8cm}
\begin{block}{Flot actuel dans G}


\usetikzlibrary{arrows}
\begin{flushleft}
\begin{center}
\begin{tikzpicture}[>=latex']
 \SetUpEdge[lw         = 1pt,
            color      = orange,
            labelcolor = red!30,
            labelstyle = {draw,sloped}]
  \tikzset{node distance = 2cm}
  \GraphInit[vstyle=Normal]
  \Vertex[x=-3, y=3]{S}
  \Vertex[x=0, y=3]{A}
  \Vertex[x=3, y=3]{D}
  \Vertex[x=3, y=0]{T}
  \Vertex[x=0, y=0]{C}
  \Vertex[x=-3, y=0]{B}
  \tikzset{EdgeStyle/.style={->}}
  \Edge[label=${2,3}$](S)(B)
  \Edge[label=${2,2}$](S)(C)
  \Edge[label=${0,3}$](S)(A)
  \Edge[label=${2,2}$](C)(T)
  \Edge[label=${0,1}$](D)(T)
  \Edge[label=${0,4}$](A)(D)
  \Edge[label=${0,1}$](C)(A)
  \Edge[label=${0,1}$](C)(D)
  \Edge[label=${0,1}$](B)(C)
  \tikzset{EdgeStyle/.style={->}, bend right}
  \Edge[label=${2,2}$](B)(T)
\end{tikzpicture}
\end{center}
\end{flushleft}

\end{block} 
\end{column}
  
  \begin{column}{3cm}
  \begin{block}{Rappel :}
    $\Delta = 1$  \\
    $F = 4$\\
  \end{block}
  \end{column}
  \end{columns}  
}

\frame{
\frametitle{Algo Capacity Scaling}
   \begin{columns}[c]
\begin{column}{8cm}
\begin{block}{G(f,1)}


\usetikzlibrary{arrows}
\begin{flushleft}
\begin{center}
\begin{tikzpicture}[>=latex']
 \SetUpEdge[lw         = 1pt,
            color      = orange,
            labelcolor = red!30,
            labelstyle = {draw,sloped}]
  \tikzset{node distance = 2cm}
  \GraphInit[vstyle=Normal]
  \Vertex[x=-3, y=3]{S}
  \Vertex[x=0, y=3]{A}
  \Vertex[x=3, y=3]{D}
  \Vertex[x=3, y=0]{T}
  \Vertex[x=0, y=0]{C}
  \Vertex[x=-3, y=0]{B}
  \tikzset{EdgeStyle/.style={->}}
  \Edge[label=$2$](C)(S)
  \Edge[label=$2$](T)(C)
  \Edge[label=$3$,color=green,labelcolor=green!30](S)(A)
  \Edge[label=$4$,color=green,labelcolor=green!30](A)(D)
  \Edge[label=$1$,color=green,labelcolor=green!30](D)(T)
  \Edge[label=$1$](C)(A)
  \Edge[label=$1$](C)(D)
  \Edge[label=$1$](B)(C)
  \Edge[label=$2$](B)(S)
  \tikzset{EdgeStyle/.style={->}, bend left}
  \Edge[label=$2$](T)(B)
\end{tikzpicture}
\end{center}
\end{flushleft}

\end{block} 
\end{column}
  
  \begin{column}{3cm}
  \begin{block}{Mise � jour du graphe d'�cart et Augmentation}
  Premier chemin :\\
$s \rightarrow a \rightarrow d \rightarrow t$  \\
$\delta = 1$  \\
$F = 5$\\
  \end{block}
  \end{column}
  \end{columns}  
}

\frame{
\frametitle{Algo Capacity Scaling}
   \begin{columns}[c]
\begin{column}{8cm}
\begin{block}{G(f,1)}


\usetikzlibrary{arrows}
\begin{flushleft}
\begin{center}
\begin{tikzpicture}[>=latex']
 \SetUpEdge[lw         = 1pt,
            color      = orange,
            labelcolor = red!30,
            labelstyle = {draw,sloped}]
  \tikzset{node distance = 2cm}
  \GraphInit[vstyle=Normal]
  \Vertex[x=-3, y=3]{S}
  \Vertex[x=0, y=3]{A}
  \Vertex[x=3, y=3]{D}
  \Vertex[x=3, y=0]{T}
  \Vertex[x=0, y=0]{C}
  \Vertex[x=-3, y=0]{B}
  \tikzset{EdgeStyle/.style={->}}
  \Edge[label=$2$](C)(S)
  \Edge[label=$2$](T)(C)
  \Edge[label=$1$](T)(D)
  \Edge[label=$1$](C)(A)
  \Edge[label=$1$](C)(D)
  \Edge[label=$1$](B)(C)
  \tikzset{EdgeStyle/.style={->}, bend left}
  \Edge[label=$2$](T)(B)
  \Edge[label=$2$](B)(S)
  \Edge[label=$2$](S)(A)
  \Edge[label=$3$](A)(D)
  \Edge[label=$1$](A)(S)
  \Edge[label=$1$](D)(A)
\end{tikzpicture}
\end{center}
\end{flushleft}

\end{block} 
\end{column}
  
  \begin{column}{3cm}
  \begin{block}{Mise � jour du graphe d'�cart et Augmentation}
 Pas d'autre chemin dans le graphe. L'algorithme est termin�.\\\
  \end{block}
  \end{column}
  \end{columns}  
}

\frame{
\frametitle{Algo Capacity Scaling}
   \begin{columns}[c]
\begin{column}{8cm}
\begin{block}{Flot maximum)}


\usetikzlibrary{arrows}
\begin{flushleft}
\begin{center}
\begin{tikzpicture}[>=latex']
 \SetUpEdge[lw         = 1pt,
            color      = orange,
            labelcolor = red!30,
            labelstyle = {draw,sloped}]
  \tikzset{node distance = 2cm}
  \GraphInit[vstyle=Normal]
  \Vertex[x=-3, y=3]{S}
  \Vertex[x=0, y=3]{A}
  \Vertex[x=3, y=3]{D}
  \Vertex[x=3, y=0]{T}
  \Vertex[x=0, y=0]{C}
  \Vertex[x=-3, y=0]{B}
  \tikzset{EdgeStyle/.style={->}}
  \Edge[label=${2,3}$,color=blue,labelcolor=blue!30](S)(B)
  \Edge[label=${2,2}$,color=blue,labelcolor=blue!30](S)(C)
  \Edge[label=${1,3}$,color=blue,labelcolor=blue!30](S)(A)
  \Edge[label=${2,2}$,color=blue,labelcolor=blue!30](C)(T)
  \Edge[label=${1,1}$,color=blue,labelcolor=blue!30](D)(T)
  \Edge[label=${1,4}$,color=blue,labelcolor=blue!30](A)(D)
  \Edge[label=${0,1}$](C)(A)
  \Edge[label=${0,1}$](C)(D)
  \Edge[label=${0,1}$](B)(C)
  \tikzset{EdgeStyle/.style={->}, bend right}
  \Edge[label=${2,2}$,color=blue,labelcolor=blue!30](B)(T)
\end{tikzpicture}
\end{center}
\end{flushleft}

\end{block} 
\end{column}
  
  \begin{column}{3cm}
  \begin{block}{Valeur du flot}
 F=5;\\
  \end{block}
  \end{column}
  \end{columns}  
}


%===================================================
\section[Experimentation et Performance]{Experimentation et Performance}
%===================================================
\subsection[Performance de l'ordinateur utilis�: Titine]{Performance de l'ordinateur utilis�: Titine}
\frame{
Version Ubuntu: 10.04\\
Intel(R) Pentium(R) Dual CPU T3200 @ 2.00GHZ
}
%+++++++++++++++++++++++++++++++++++++++++++++++++++
\subsection[Temps d'execution en fonction du nombre de sommets en comptant le temps de cr�ation des graphes]{Temps d'execution en fonction du nombre de sommets comptant le temps de cr�ation des graphes}
%---------------------------------------------------

%--------------------------------------------------
\frame{
\frametitle{Ford-Fulkerson}
\begin{tikzpicture}
 \begin{axis}[grid= major
 ,  width=0.8\textwidth ,xlabel = sommets ,ylabel = temps ,xmin = 0, xmax = 1100,ymin = 0, ymax = 3]
 \addplot[color=red,mark=x] coordinates {
   (200,0.05000)
   (300,0.10000)
   (400,0.35210)
   (500,0.83900)
   (600,1.23500)
   (700,2.03100)
   (800,2.46777)
   (900,2.75100)
   (1000,3.58500)
	};  
  \end{axis}
\end{tikzpicture}
}

\frame{
\frametitle{Capacity-Scaling}
\begin{tikzpicture}
 \begin{axis}[grid= major
 ,  width=0.8\textwidth ,xlabel = sommets ,ylabel = temps ,xmin = 0, xmax = 1100,ymin = 0, ymax = 3]
 \addplot[color=red,mark=x] coordinates {
   (200,0.02000)
   (300,0.06000)
   (400,0.10000)
   (500,0.17900)
   (600,0.25500)
   (700,0.36000)
   (800,0.40000)
   (900,0.60100)
   (1000,0.80000)
	};  
  \end{axis}
\end{tikzpicture}
}


\frame{
\frametitle{Edmons-Karp}
\begin{tikzpicture}
 \begin{axis}[grid= major
 ,  width=0.8\textwidth ,xlabel = sommets ,ylabel = temps ,xmin = 0, xmax = 1100,ymin = 0, ymax = 3]
 \addplot[color=red,mark=x] coordinates {
   (200,0.01500)
   (300,0.05000)
   (400,0.10000)
   (500,0.16000)
   (600,0.20000)
   (700,0.30000)
   (800,0.40000)
   (900,0.60000)
   (1000,0.75000)
	};  
  \end{axis}
\end{tikzpicture}
}


\frame{
\frametitle{Dinic}
\begin{tikzpicture}
 \begin{axis}[grid= major
 ,  width=0.8\textwidth ,xlabel = sommets ,ylabel = temps ,xmin = 0, xmax = 1100,ymin = 0, ymax = 3]
 \addplot[color=red,mark=x] coordinates {
   (200,0.40000)
   (300,1.50000)
   (400,2.90000)
   (500,5.00000)
	};  
  \end{axis}
\end{tikzpicture}
}

\frame{
\frametitle{Comparaison des algorithmes}
\begin{tikzpicture}
 \begin{axis}[grid= major
 ,  width=0.8\textwidth ,xlabel = sommets ,ylabel = temps ,xmin = 0, xmax = 1100,ymin = 0, ymax = 3]
 \addplot[color=red,mark=triangle*] coordinates {
   (200,0.40000)
   (300,1.50000)
   (400,2.90000)
   (500,5.00000)
	};  
 \addlegendentry{Dinic}
\addplot[color=blue,mark=*] coordinates {
   (200,0.01500)
   (300,0.05000)
   (400,0.10000)
   (500,0.16000)
   (600,0.20000)
   (700,0.30000)
   (800,0.40000)
   (900,0.60000)
   (1000,0.75000)
	};  
\addlegendentry{Edmons-Karp}
 \addplot[color=green,mark=x] coordinates {
   (200,0.02000)
   (300,0.06000)
   (400,0.10000)
   (500,0.17900)
   (600,0.25500)
   (700,0.36000)
   (800,0.40000)
   (900,0.60100)
   (1000,0.80000)
	};  
\addlegendentry{Capacity-Scaling}
 \addplot[color=yellow,mark=square*] coordinates {
   (200,0.05000)
   (300,0.10000)
   (400,0.35210)
   (500,0.83900)
   (600,1.23500)
   (700,2.03100)
   (800,2.46777)
   (900,2.75100)
   (1000,3.58500)
	};
 \addlegendentry{Ford-Fulkerson}
  \end{axis}
\end{tikzpicture}
}

%+++++++++++++++++++++++++++++++++++++++++++++++++++
\subsection[Evolutions � apporter]{Evolutions � apporter}
%---------------------------------------------------

%--------------------------------------------------
\frame{
\frametitle{Evolutions}
  \begin{block}{Repr�sentation du graphe}
    La repr�sentation sous forme de matrice nous a parue interr�ssante dans un premier temps, de par sa simplicit� � l'utilisation 
    mais nous avons r�alis� qu'une liste de voisins aurait �t� certainement plus repr�sentative pour les parcours de chemin.
  \end{block}

 \begin{block}{Graphe de couche}
   Le graphe de couche est peut-etre mal repr�senter car nous l'avons repr�senter comme un graphe normal, dans une matrice.\\
   Ceci gene un peu l'utilisation de ce graphe
  \end{block}
}

\frame{
\frametitle{Evolutions}

 \begin{block}{Algorithme de Dinic}
    L'algorithme de Dinic, comme les autres, fonctionne mais il manque certainement d'optimisation
  \end{block}

}




%===================================================
\section[Demonstration du fonctionnement sous TIKZ]{Demonstration du fonctionnement sous TIKZ}
%===================================================

%+++++++++++++++++++++++++++++++++++++++++++++++++++
\subsection[Demonstration du fonctionnement sous TIKZ]{Demonstration du fonctionnement sous TIKZ}
%---------------------------------------------------

%--------------------------------------------------
\frame{
\frametitle{Démo}


}

\end{document}
