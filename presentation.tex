\documentclass[pdftex]{beamer}
\usepackage{beamerthemesplit}
\usepackage[latin1]{inputenc}
\usepackage[T1]{fontenc}
%\usepackage[french]{babel}
\usepackage{times}
\usepackage{tabularx}

\mode<presentation>
{
  \usetheme{Warsaw}
  \usefonttheme[onlysmall]{structurebold}
}

\beamertemplatetransparentcovered
\beamerboxesdeclarecolorscheme{myalert}{red}{black!5!averagebackgroundcolor}
\beamerboxesdeclarecolorscheme{mybox}{blue}{black!5!averagebackgroundcolor}

\usepackage{pgf,pgfarrows,pgfnodes,pgfautomata,pgfheaps,pgfshade}


\font\bbfnt=msbm6
\def\bbN{\mbox{\bbfnt N}}
\def\bbR{\mbox{\bbfnt R}}

\DeclareGraphicsExtensions{.pdf,.pdftex_t,.png,.jpg,.gif} 
\graphicspath{{./}{../Images/}} 

%===================================================
%===================================================

\title{Projet pratique Algorithme / Complexit� / Calculabilit�}
\author{Jean-Marc Lagniez, Viktor Lesnyak, Pierre-Alexandre Cimbe, Ahmed Rafik}
\institute{Master Informatique - Universit� Montpellier II}
\date{2013}

%===================================================
%===================================================

\begin{document}

\frame{\titlepage}

%===================================================
\section<presentation>*{Plan}
%===================================================

\note{L'exops� est compos� de trois parties, une partie pr�senation des algorithmes utilis�s,
 une partie exp�rimentation et performances et une partie d�monstration du programme}

%----------------------------------------
\begin{frame}
  \frametitle{Plan}
  \tableofcontents[part=1,pausesections]
\end{frame}

%----------------------------------------
% cette partie entraine l'affichage du plan � chaque nouveau d�but de section
\AtBeginSection[]
{
  \begin{frame}<beamer>
    \frametitle{Plan}
    \tableofcontents[current,currentsection]
  \end{frame}
}


%=================================================
\part<presentation>{Les Algorithmes utilis�s}
%===================================================
\section[Les algorithmes �tudi�s]{Les algorithmes �tudi�s}
%===================================================

%+++++++++++++++++++++++++++++++++++++++++++++++++++
\subsection[Ford-Fulkerson]{Ford-Fulkerson}
%---------------------------------------------------

%--------------------------------------------------
\frame{
\frametitle{AlgoFF}


}


%--------------------------------------------------
\frame{
\frametitle{AlgoFF-Suite}

}

%+++++++++++++++++++++++++++++++++++++++++++++++++++
\subsection[Edmonds-Karp]{Edmonds-Karp}
%---------------------------------------------------
%--------------------------------------------------
\frame{
\frametitle{AlgoEK}
   
}

%+++++++++++++++++++++++++++++++++++++++++++++++++++

\subsection[Dinic]{Dinic}
%---------------------------------------------------
%--------------------------------------------------
\frame{
\frametitle{AlgoD}
   
}

%+++++++++++++++++++++++++++++++++++++++++++++++++++

\subsection[Cacity Scaling]{Capacity Scaling}
%---------------------------------------------------
%--------------------------------------------------
\frame{
\frametitle{AlgoCS}
   
}

%===================================================
\section[Experimentation et Performance]{Experimentation et Performance}
%===================================================

%+++++++++++++++++++++++++++++++++++++++++++++++++++
\subsection[Temps d'execution en fonction du nombre de sommets]{Temps d'execution en fonction du nombre de sommets}
%---------------------------------------------------

%--------------------------------------------------
\frame{
\frametitle{???????}


}


%+++++++++++++++++++++++++++++++++++++++++++++++++++
\subsection[Temps d'execution en fonction de la capacit� maximale]{Temps d'execution en fonction de la capacit� maximale}
%---------------------------------------------------

%--------------------------------------------------
\frame{
\frametitle{????bis?????}

}

%+++++++++++++++++++++++++++++++++++++++++++++++++++
\subsection[Espace Memoire utilis�]{Espace Memoire utilise}
%---------------------------------------------------

%--------------------------------------------------

\frame{
  \frametitle{???}
  
}	



%===================================================
\section[Demonstration du fonctionnement sous TIKZ]{Demonstration du fonctionnement sous TIKZ}
%===================================================

%+++++++++++++++++++++++++++++++++++++++++++++++++++
\subsection[Demonstration du fonctionnement sous TIKZ]{Demonstration du fonctionnement sous TIKZ}
%---------------------------------------------------

%--------------------------------------------------
\frame{
\frametitle{D�mo}


}

\end{document}
