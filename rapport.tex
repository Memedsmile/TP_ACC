\documentclass{article}
\usepackage[utf8]{inputenc}
\usepackage[T1]{fontenc}
\usepackage{hyperref}
\usepackage[english]{babel}
\usepackage{calc}
\usepackage{amsmath}
\usepackage{amssymb}
\usepackage{mathrsfs}
\usepackage{amsthm}

\title{TP d'Algo/Complexité/Calculabilité}
\author{
  CIMBE Pierre-Alexandre \\
  LAGNIEZ Jean-Marc \\
  LESNYAK Viktor \\
  RAFIK Ahmed}
\date\today

\begin{document}

\maketitle

\section{Partie théorique}
\subsection{Partie algorithmique}
\subsection{Partie complexité}

\subsubsection{Exercie 5}

\begin{enumerate}
\item \begin{enumerate}
  \item
    SAT : Un  problème SAT est un problème de décision visant à montrer l'existence d'une interprétation satisfaisant un ensemble de variable propositionnelle. (Formule logique CNF)
    3-SAT : Cas particulier du problème SAT dans lequel les clauses sont toutes de taille 3.

  \item
    On dit qu'il existe une réduction d'un problème P à un problème P' s'il existe une fonction f telle que x $\in$ D(P) <=> f(x) $\in$ D(P')

  \item
    3-SAT est un cas particulier de SAT, or SAT $\in$ NP.
    Donc 3-SAT $\in$ NP

    SAT $\in$ NP-Complet

    Nous allons chercher à réduire un problème Sat à un problème 3-SAT : 

    Soit P une instance du problème SAT.
    
    $\twoheadrightarrow$Soit U={$l_1vl_2vl_3......vl_k$} une clause de taille k > 3

    On la divise en 2 clause :
    $\succ$ une clause de taille $\lfloor$k/2$\rfloor$+1 en complétant par une variable x$\notin$U 
    $\succ$ une clause de taille $\lceil$k/2$\rceil$+1 en complétant par $\overline x$ le complémentaire de x.

    On applique ce principe récursivement jusqu'à obtenir des clauses de taille 3

    $\twoheadrightarrow$Soit U={$l_1vl_2$} une clause de taille 2
    
    On clone la clause U pour avoir 2 clauses $U_1$ et $U_2$ auxquelles on ajoute respectivement une variable x et son complémentaire $\overline x$
    on obtient :  $U_1$={$l_1vl_2$vx}
    $U_2$={$l_1vl_2$v$\overline x$}

    $\twoheadrightarrow$Soit U={$l_1$} une clause de taille 1
    On force $l_1$ à vrai et on retire les clauses unitaire.

    On obtient ainsi un problème P' de type 3-SAT.    

    Donc 3-SAT est NP-Complet.

    \item
      Soit $l_1$, $l_2$, $l_3$, $l_4$ une clause de taille 4

      $l_1, l_2, l_3, l_4\left\lbrace 
      \begin{array}{lcl} 
        l_1 v l_2 v u\\
        l_3 v l_4 v \overline u
      \end{array}\right.$
    
\end{enumerate}
\end{enumerate}



\subsection{Partie calculabilité}


\subsubsection{Exercie 7}

\noindent\fbox{\parbox{\linewidth-2\fboxrule-2\fboxsep}{ \begin {enumerate} \scriptsize  
\item \textbf{Comment enumérer les couples d'entiers?}\\
\item \textbf{Donner les fonctions de codage et de décodage f\oldstylenums{1} $\rightarrow$ x et f\oldstylenums{2} $\rightarrow$ y}\\
\item \textbf{Montrer que l'on peut coder les triplets. Géneraliser aux k-uplets.}\\
\item \textbf{Pensez-vous que l'on peut coder les éléments de l'intervalle [0,1]. Justifier.}
\end{enumerate} }}\\\\

\begin{enumerate}
\item Soit (x,y) $\in$ $\mathbb{N}$ * $\mathbb{N}$, alors faire x + y et trié par ordre lexicographique
\item La fonction de codage est : \[z = \frac{(x+y)(x+y+1)}{2} + y\]\\
Pour les fonction de décodage, posons t tel que \[t = x + y\]
On va prendre t tel que si t augmente de 1 alors \[\frac{t(t+1)}{2} > z\] sinon on a \[\frac{t(t+1)}{2} \le z\]
La fonction de décodage de y est: \[z =\frac{t(t+1)}{2} + y\] \[y = z - \frac{t(t+1)}{2}\]
La fonction de décodage de x est: \[x = t - y\] \[x = -z + t + \frac{t(t+1)}{2}\] \[x = -z + \frac{t(t+3)}{2}\]
\item Pour coder les triplets, il suffit de coder deux entier et coder le résultat et le dernier entier. 
  \[h(x,y,z)=c(x , c(y,z))\]
On peut repeter se raisonement pour les k-uplets, ainsi on a 
\[k(x\oldstylenums{1},x\oldstylenums{2}... x\oldstylenums{k}) = c(x\oldstylenums{1} , c(x\oldstylenums{2} , ... c(x\oldstylenums{k-1},x\oldstylenums{k}))\]
\item On ne peut pas coder les éléments de l'intervalle [0,1] car l'ensemble n'est pas dénombrable. On utilise la diagonal de cantor sur cette ensemble.\\
Supposons que l'on puisse numeroter $\mathbb{N}$  $\rightarrow$ [0,1] et on en définie la suite S telle que tout éléments de [0,1] soit élément de la suite S.
Et on définie un réel r tel que la partie entière est égal à 0 et que chaque décimal en position n est égal à sn(n)\footnote{la nème décimal du nème élément de S}+1 si sn(n) est différent de 9 et sn(n)-1 si sn(n) est égal à 9.\\
Par construction, r n'est pas dans S sinon on aurait un Sn tel que \[Sn(n)=r(n)=Sn(n)+1\] ou \[Sn(n)=r(n)=Sn(n)-1\] C'est absurbe, ainsi ce n'est pas dénombrable. 
   
  
\end{enumerate}




\subsubsection{Exercice 8}
\begin{enumerate}
\item Les fonctions primitives récursives sont toutes les fonctions que l'on peut construire à partir des fonctions de base pas composition et récursion primitive.\\

  Exemple\\ \\
Soit les fonctions primitives: \\
O $\in \mathbb{N}^0$, $\pi_i^k \in \mathbb{N}^k$ et SUC $\mathbb{N}^1$ 

 \[O() = 0 \] 
\[\pi_i^k(x_1,x_2...,x_k) = x_i \] 
\[SUC(x_1) = x_1 + 1 \] 

Soit la fonction qu'on utilise pour la récursion primitive:\\
g $\in \mathbb{N}^1$
\[g() = SUC(O()) \]

Soit la fonction recursive primitive:\\
f $\in \mathbb{N}^1$
\[f(0) = g()\]
\[f(SUC(n)) = \pi_1^2(f(n),n)\]
  
\item  yoloooooooo je ne sais pas 
\item \begin{enumerate}\item Soit la fonction somme défini ainsi:\\
 Sum $\in \mathbb{N}^2$ 
\[Sum(0,y) = \pi_1^1(y) = y \]
\[Sum(Suc(x),y) = \pi_2^3(x,Sum(x,y),y)\]
\item Soit la fonction Mult défini ainsi: \\
Mult $\in \mathbb{N}^2$
\[Mult(O,y) = 0() = 0\]
\[Mult(1,y) = \pi_1^1(y) = y\]
\[Mult(Suc(x),y) = \pi_2^3(x,Sum(Mult(x,y),y),y)\]
\item Soit la fonction puissance défini aisni:\\
 $X^Y$ $\in \mathbb{N}^2$
\[X^Y(x,0) = Suc(0()) = 1\]
\[X^Y(x,Suc(y)) = \pi_2^3(x,Mult(X^Y(x,y),x),y)\]
\item Soit la fonction prédecesseurs tel que: \\
Pred $\in \mathbb{N}^1$
\[Pred(0) = O() = 0\]
\[Pred(Suc(x)) = \pi_1^2(x,Pred(x))\]
\item Soit la fonction soustraction tel que:\\
X-Y $\in \mathbb{N}^2$
\[X-Y(0,y) = 0() = 0\]
\[X-Y(x,0) = \pi_1^1(x) = x\]
\[X-Y(x,y) = \pi_2^3(x,X-Y(Pred(x),Pred(y)),y))\]
\item Soit la fonction sg tel que:
sg $\in \mathbb{N}^1$
\[sg(0) = 0() = 0\]
\[sg(Suc(x)) = \pi_1^2(1,Suc(x))\]
\item Soit la fonction X > Y tel que :\\
X>Y $\in \mathbb{N}^2$
\[X>Y(0,y) = 0\]
\[X>Y(x,0) = 1\]
\[X>Y(x,y) = \pi_2^3(x,X>Y(Pred(x),Pred(y)),y)\]
Soit la fonction X $\ge$ Y tel que :\\
X$\ge$Y $\in \mathbb{N}^2$
\[X\ge Y(0,0) = 1\]
\[X\ge Y(0,y) = 0\]
\[X\ge Y(x,0) = 1\]
\[X\ge Y(x,y) = \pi_2^3(x,X>Y(Pred(x),Pred(y)),y)\]

\end{enumerate}
\item \begin{enumerate} 

\item Voici la fonction d'Ackerman pour 0 $\le$ m $\le$ 3 et 0 $\le$ n $\le$ 4 \\
\begin{center}
\begin{tabular}{| c || c | c | c | c | c |}
\hline
m/n & 0 & 1 & 2 & 3 & 4 \\
\hline
\hline
0 & 1 & 2 & 3 & 4 & 5 \\
\hline
1 & 2 & 3 & 4 & 5 & 6 \\
\hline
2 & 3 & 5 & 7 & 9 & 11 \\
\hline
3 & 5 & 13 & 28 & 58 & 118 \\
\hline
\end{tabular}
\end{center}

\item Fesons une preuve par récurrence
\[A_0(n) = Suc(n) = n + 1\]
Hypothèse: A\tiny m\normalsize (n) est primitive récursive\\
Montrons que A\tiny m+1\normalsize (n) est primitive récursive\\ 

Si n = 0, on a que A\tiny m+1\normalsize (n) = A\tiny m\normalsize (1). D'après l'hypothèse de réccurence, on a que A\tiny m\normalsize (n) est primitive récursive. Donc A\tiny m+1\normalsize (n) est primitf recursive\\

Si n > 0, on a que A\tiny m+1\normalsize (n) = A\tiny m\normalsize (A\tiny m+1\normalsize (n)).\\
Posons n' = A\tiny m+1\normalsize (n). Donc on a A\tiny m\normalsize (n'). D'après l'hypothèse de réccurence, on a que A\tiny m\normalsize (n) est primitive récursive pour tous n $\in \mathbb{N}$ . Donc A\tiny m+1\normalsize (n) est primitif recursive.\\

\item 
Fesons une preuve par récurrence 
\[A_0(n) = n+1 \] 
n+1 > n donc c'est vrai au premier rang\\
Hypothèse: A\tiny m\normalsize (n) > n \\
Montrons que A\tiny m+1\normalsize (n) > n\\ \\
Maitenant, on applique une récurrence sur n\\
n = 0 : A\tiny m+1\normalsize (1) > 1  > 0
Hypothèse: A\tiny m+1\normalsize (n) > n \\
Montrons que: A\tiny m+1\normalsize (n+1) > n+1\\
On utilise les deux hypothèse de réccurence:\\
\[A\tiny m+1\normalsize (n+1) = A\tiny m\normalsize (A\tiny m+1\normalsize (n)) > A\tiny m+1\normalsize (n) > n \]
Ainsi \[A\tiny m+1\normalsize (n) \ge n+1\]
Donc \[A\tiny m+1\normalsize (n+1) > n+1\]
On peux donc conclure que A\tiny m\normalsize (n) > n\\

\item  Il faut montrer que A\tiny m+1\normalsize (n) - A\tiny m\normalsize (n)  $\ge$ 0\\
Fesons une preuve par récurrence sur m \\
\[A_0(n+1) - A_0(n) = n + 1 - n = 1\] 
Hypothèse: A\tiny m\normalsize (n+1) - A\tiny m\normalsize (n) > 0\\
Montrons que A\tiny m+1\normalsize (n+1) - A\tiny m+1\normalsize (n) > 0\\
A\tiny m+1\normalsize (n+1) = A\tiny m\normalsize (A\tiny m+1\normalsize (n)) > A\tiny m\normalsize (n)\\
On peut conclure que\\ 
A\tiny m\normalsize (n+1) - A\tiny m\normalsize (n) > 0

\item 
Pour n = 0: A\tiny m+1\normalsize (0) = A\tiny m\normalsize (1). De plus, d'après la question précédente, on a que A\tiny m\normalsize (1) > A\tiny m\normalsize (0)\\\\
Pour n > 0: A\tiny m+1\normalsize (n) = A\tiny m\normalsize (A\tiny m+1\normalsize (n-1)). De plus on a que A\tiny m\normalsize (n-1) > n - 1 $\rightarrow$ A\tiny m\normalsize (n-1) $\ge$ n\\
Comme la fonction est strictement croissante, on a que  A\tiny m\normalsize (A\tiny m\normalsize (n-1)) $\ge$ A\tiny m\normalsize (n) \\ \\
On peux en conclure que \\
A\tiny m+1\normalsize (n) = A\tiny m\normalsize (A\tiny m+1\normalsize (n-1)) $\ge$ A\tiny m\normalsize (n) 

\item D'après les question précédente, on a montré que A\tiny m+1\normalsize (n) $\ge$ A\tiny m\normalsize (n) et que A\tiny m\normalsize (n+1) > A\tiny m\normalsize (n). Ceci prouve que $A_m^k$ est strictement croissante. 

\item Fesons une preuve pas récurrence sur k.\\
Au cas de base, on a bien $A_{m+1}$\normalsize (n) $\ge$ $A_m$\normalsize (n)\\
Hypothèse: $A_{m+1}$\normalsize (n + k) $\ge$ $A^k_m$\normalsize (n)\\
Montrons que : $A_{m+1}$(n + k + 1) $\ge$ $A^{k+1}_m$(n)\\
D'après l'hypothèse de réccurence, on a \\
\[A^{k+1}_m = A_m(A^k_m(n)) \le A_m(A_{m+1}(n + k))\] 
De plus:\\
\[A_{m+1}(n + k + 1) = A_m(A_{m+1}(n + k))\]
On peut conclure que:\\
\[A^{k+1}_m = A_m(A^k_m(n)) \le A_{m+1}(n + k + 1)\]
\item 
Fesons une preuve par l'absurbe, soit la fonction d'Ackermann primitive récursive.\\\\
Sois la fonction \[f: \mathbb{N} \rightarrow \mathbb{N} : n \rightarrow A(n,2n)\]
Comme la fonction d'Ackerman est primitive récursive alors f est primitive récursive.   
\end{enumerate}
\end{enumerate}
\end{document}
