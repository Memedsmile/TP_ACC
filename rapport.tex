\documentclass{article}
\usepackage[utf8]{inputenc}
\usepackage[T1]{fontenc}
\usepackage{hyperref}
\usepackage[english]{babel}
\usepackage{calc}
\usepackage{amsmath}
\usepackage{amssymb}
\usepackage{mathrsfs}
\usepackage{amsthm}

\title{TP d'Algo/Complexité/Calculabilité}
\author{
  CIMBE Pierre-Alexandre \\
  LAGNIEZ Jean-Marc \\
  LESNYAK Viktor \\
  RAFIK Ahmed}
\date\today

\begin{document}

\maketitle

\section{Partie théorique}
\subsection{Partie algorithmique}
\subsection{Partie complexité}
\subsection{Partie calculabilité}


\subsubsection{Exercie 7}

\noindent\fbox{\parbox{\linewidth-2\fboxrule-2\fboxsep}{ \begin {enumerate} \scriptsize  
\item \textbf{Comment enumérer les couples d'entiers?}\\
\item \textbf{Donner les fonctions de codage et de décodage f\oldstylenums{1} $\rightarrow$ x et f\oldstylenums{2} $\rightarrow$ y}\\
\item \textbf{Montrer que l'on peut coder les triplets. Géneraliser aux k-uplets.}\\
\item \textbf{Pensez-vous que l'on peut coder les éléments de l'intervalle [0,1]. Justifier.}
\end{enumerate} }}\\\\

\begin{enumerate}
\item Soit (x,y) $\in$ $\mathbb{N}$ * $\mathbb{N}$, alors faire x + y et trié par ordre lexicographique
\item La fonction de codage est : \[z = \frac{(x+y)(x+y+1)}{2} + y\]\\
Pour les fonction de décodage, posons t tel que \[t = x + y\]
La fonction de décodage de y est: \[z =\frac{t(t+1)}{2} + y\] \[y = z - \frac{t(t+1)}{2}\]
La fonction de décodage de x est: \[x = t - y\] \[x = -z + t + \frac{t(t+1)}{2}\] \[x = -z + \frac{t(t+3)}{2}\]
\item Pour coder les triplets, il suffit de coder deux entier et coder le résultat et le dernier entier. 
  \[h(x,y,z)=c(x , c(y,z))\]
On peut repeter se raisonement pour les k-uplets, ainsi on a 
\[k(x\oldstylenums{1},x\oldstylenums{2}... x\oldstylenums{k}) = c(x\oldstylenums{1} , c(x\oldstylenums{2} , ... c(x\oldstylenums{k-1},x\oldstylenums{k}))\]
\item On ne peut pas coder les éléments de l'intervalle [0,1] car l'ensemble n'est pas dénombrable. On utilise la diagonal de cantor sur cette ensemble.\\
Supposons que l'on puisse numeroter $\mathbb{N}$  $\rightarrow$ [0,1] et on en définie la suite S telle que tout éléments de [0,1] soit élément de la suite S.
Et on définie un réel r tel que la partie entière est égal à 0 et que chaque décimal en position n est égal à sn(n)\footnote{la nème décimal du nème élément de S}+1 si sn(n) est différent de 9 et sn(n)-1 si sn(n) est égal à 9.\\
Par construction, r n'est pas dans S sinon on aurait un Sn tel que \[Sn(n)=r(n)=Sn(n)+1\] ou \[Sn(n)=r(n)=Sn(n)-1\] C'est absurbe, ainsi ce n'est pas dénombrable. 
   
  
\end{enumerate}




\subsubsection{Exercice 8}
\begin{enumerate}
\item Les fonctions primitives récursives sont toutes les fonctions que l'on peut construire à partir des fonctions de base pas composition et récursion primitive.\\

  Exemple\\ \\
Soit les fonctions primitives: \\
O $\in \mathbb{N}^0$, $\pi_i^k \in \mathbb{N}^k$ et SUC $\mathbb{N}^1$ 

 \[O() = 0 \] 
\[\pi_i^k(x_1,x_2...,x_k) = x_i \] 
\[SUC(x_1) = x_1 + 1 \] 

  
\item  
\end{enumerate}
\end{document}
